% Changing book to article will make the footers match on each page,
% rather than alternate every other.
%
% Note that the article class does not have chapters.
%\documentclass[letterpaper,10pt,twoside,twocolumn,openany]{book}
\documentclass[letterpaper,12pt,openany]{extbook}

\usepackage[francais]{babel}

\usepackage[utf8]{inputenc}
\usepackage{lipsum}
\usepackage{listings}
\usepackage[hidelinks]{hyperref}


\usepackage{dnd}

\lstset{%
  basicstyle=\ttfamily,
  language=[LaTeX]{TeX},
}

% Start document
\begin{document}
\AddToShipoutPicture{\BackgroundPic}

\begin{titlepage} % Suppresses displaying the page number on the title page and the subsequent page counts as page 1

  \raggedleft % Right align the title page

  \rule{1pt}{\textheight} % Vertical line
  \hspace{0.05\textwidth} % Whitespace between the vertical line and title page text
  \parbox[b]{0.75\textwidth}{ % Paragraph box for holding the title page text, adjust the width to move the title page left or right on the page

    {\color{titlered}\Huge\bfseries Manuel de désamorçage}\\[2\baselineskip] % Title
    {\large\textit{À manier avec précaution}}\\[4\baselineskip] % Subtitle or further description
    {
      \textsc{Charles Paulet \\[0.5\baselineskip]
        Léo Paol \\[0.5\baselineskip]
        Théophile Champion \\[0.5\baselineskip]
        Nicolas-Emmanuel Robert
      }
    }

    \vspace{0.5\textheight} % Whitespace between the title block and the publisher

    {\noindent \url{https://github.com/valkheim/KTNE-IRL} }\\[\baselineskip]
  }

\end{titlepage}
% Your content goes here

% Comment this out if you're using the article class.
\chapter{Chapter 1: The Dark \LaTeX}

\section{Main Section}
\lipsum[2] % filler text

\begin{quotebox}
  Ceci est une \lstinline!quotebox!, elle sert à citer quelque chose mais
  c'est aussi un joli ruban utilisable un peu partout. 
\end{quotebox}

\subsection{Subsection}
\subsubsection{subsubsection}

\begin{commentbox}{Commentbox}
  Je suis une \lstinline!commentbox!. Je suis une \lstinline!paperbox! dans 
\end{commentbox}

\lipsum[3]

\begin{figure}[!t]
  \begin{paperbox}{Behold, the Paperbox!}
    Je suis une \lstinline!paperbox! . Je suis flottant dans un
  coin de page. Dans l'idéal, je suis utilisé avec un environnement figure.
  \end{paperbox}
\end{figure}

% For more columns, you can say \begin{dndtable}[your options here].
% For instance, if you wanted three columns, you could say
% \begin{dndtable}[XXX]. The usual host of tabular parameters are
% available as well.
  \header{Joli tableau}
  \begin{dndtable}
    \textbf{head}  & \textbf{head} \\
    value  & value \\
    value  & value \\
    value  & value
  \end{dndtable}

  \lipsum[2]

  % You can optionally not include the background by saying
  % begin{moduleboxnobg}
  \begin{modulebox}{Module}
    \begin{hangingpar}
      \textit{Exemple de module}
    \end{hangingpar}
    \modulesection{Description}
    \begin{moduleaction}[Difficulté]
      Vraiment très facile.
    \end{moduleaction}
    \hline%
    \modulesection{Composants :}
    \begin{moduleaction}[Bouton]
      J'ai un bouton connecté sur le pin 4.
    \end{moduleaction}

    \begin{moduleaction}[Leds]
      Mes leds rouges et vertes indiquent si je suis actif ou non.
    \end{moduleaction}
  \end{modulebox}

  \section{Colors}

  Ce paquet donne accès à pas mal de couleurs pour styliser les
  \lstinline!commentbox!, \lstinline!quotebox!, \lstinline!paperbox!, et les
  \lstinline!dndtable!.

  \begin{dndtable}[lX]
    \textbf{Color}         & \textbf{Description} \\
    \lstinline!commentboxcolor! &  \lstinline!commentbox! background. \\
    \lstinline!paperboxcolor!   &  \lstinline!paperbox! background. \\
    \lstinline!quoteboxcolor!   &  \lstinline!quotebox! background. \\
    \lstinline!tablecolor!      &  background des lignes \lstinline!dndtable! paires. \\
  \end{dndtable}

  Voir tableau~\ref{tab:colors} pour une liste des couleurs.

  \begin{table*}
    \begin{dndtable}[XX]
      \textbf{Color}
      \lstinline!PhbLightGreen!
      \lstinline!PhbLightCyan!
      \lstinline!PhbMauve!
      \lstinline!PhbTan!
      \lstinline!DmgLavender!
      \lstinline!DmgCoral!
      \lstinline!DmgSlateGray! (\lstinline!DmgSlateGrey!)
      \lstinline!DmgLilac!
    \end{dndtable}
    \caption{Couleurs du package}%
    \label{tab:colors}
  \end{table*}

  \begin{itemize}
    \item Utilisez \lstinline!\setthemecolor[<color>]! pour changer la couleurs
      de \lstinline!themecolor!, \lstinline!commentcolor!,
      \lstinline!paperboxcolor!, ou \lstinline!tablecolor!.
    \item \lstinline!\setthemecolor! sans argument met la couleur du
      \lstinline!themecolor! actuel.
    \item \lstinline!commentbox!, \lstinline!dndtable!, \lstinline!paperbox!, et
      \lstinline!quoteboxcolor! acceptent aussi un argument de couleur optionnel.
  \end{itemize}

\subsection{Exemples}

\subsubsection{Utiliser \lstinline!themecolor! avec \lstinline!PhbMauve!}

\begin{lstlisting}
\setthemecolor[PhbMauve]

\begin{paperbox}{Exemple}
  \lipsum[2]
\end{paperbox}

\setthemecolor[PhbLightCyan]

\header{Exemple}
\begin{dndtable}[cX]
  \textbf{d2} & \textbf{Item} \\
  1           & Super projet \\
  2           & Beau manuel \\
\end{dndtable}
\end{lstlisting}

\begingroup
\setthemecolor[PhbMauve]

\begin{paperbox}{Exemple}
  \lipsum[2]
\end{paperbox}

\setthemecolor[PhbLightCyan]

\header{Exemple}
\begin{dndtable}[cX]
  \textbf{d2} & \textbf{Item} \\
  1           & Super projet \\
  2           & Beau manuel \\
\end{dndtable}
\endgroup

\subsubsection{Utiliser les arguments de couleur}

\begin{lstlisting}
\begin{dndtable}[cX][DmgCoral]
  \textbf{d2} & \textbf{Item} \\
  1           & Super projet \\
  2           & Beau manuel \\
\end{dndtable}
\end{lstlisting}

\begin{dndtable}[cX][DmgCoral]
  \textbf{d2} & \textbf{Item} \\
  1           & Super projet \\
  2           & Beau manuel \\
\end{dndtable}

% End document
\end{document}
