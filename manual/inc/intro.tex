\chapter{Soyez le ou la bienvenu·e}
Pas de chance, vous voici avec la vie de votre ami\md e entre les mains. Ce sont
des choses qui arrivent me direz-vous. Quand bien même on ne l'aurait pas
souhaité c'est un juste retour des choses pour vous de vous retrouver dans cette
situation. Ca vous apprendra à aller fouiller le laboratoire de la société du
mal des ingénieur\md e\md s maléfiques\dots  Maintenant il vous reste à faire de
votre mieux pour ne pas que tout parte en fumée ! Vous trouverez dans ce manuel
toutes les informations (et sûrement beaucoup plus, voire même un peu trop) qui
vous permettrons de désamorçer la bombe, même des plus terribles. Ne négligez
aucun détail, ce serait dommage\dots Mais surtout bonne chance !\par

\section{Petit cours de désamorçage}
Une bombe explose quand le compteur tombe à 00:00. La seule option pour la
désamorcer est de désamorcer chacun de ses modules. Et ils sont nombreux.
Vous devriez pouvoir les discerner, ils sont imbriqués dans la malette.
\subsection{Modules}
Chaque bombe contient onze modules qui doivent être désamorcés. Ils peuvent être
désamorcés dans l'ordre souhaité et normalement ils ne s'influcent pas les uns
les autres. Mais qui sait ce qui se passe dans la tête d'ingénieur\md e\md s
fous\dots Enfin vous verrez bien sur quoi vous tomberez !\par
Dans la suite du manuel vous trouverez les instructions pour désamorcer chacun
des modules. Lisez les instructions attentivement et coordonnez-vous bien avec
votre équipier·ère ou vous n'en ressortiez pas vivant.\par
\subsection{Pénalités}
Eh oui, des pénalités ! Les ingénieur\md e\md s d'EvilTek sont vicieux et se
doutaient bien que des petit\md e\md s rusé\md e\md s comme vous iraient s'amuser
à bidouiller des engins explosifs. C'est pour ça qu'à chaque tentative de
désamorçage infructueuse d'un module, vous subirez des pénalités de temps. Elles
seront d'autant plus importantes que le niveau de difficulté est élevé. Le temps
file mais ne courrez pas plus vite à votre perte, tâchez de réfléchir et de
prendre les bonnes décisions.\par
\subsection{Fouinez}
Les informations ne sont pas forcément là où on les attend. Des numéros, textes,
et dessins disséminés sur la bombe et le manuel pourrait vous être utiles.
Pensez à vérifier l'agencement des modules ou encore votre propre téléphone !
Qui sait ce qu'on pu imaginer les constructeur\md trice\md s\dots
