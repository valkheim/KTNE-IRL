\subsection{Wire Warrior}
Sous ce nom terrible se cacherait-il simplement des fils à brancher, débrancher
et rebrancher correctement ? On l'espère\dots
\vspace{.5cm}
\begin{modulebox}{/wire/warrior -help}
  \begin{hangingpar}
    \textit{Le fil rouge sur le bouton rouge, le fil vert\dots}
  \end{hangingpar}
  \modulesection{Description}
  \begin{moduleaction}[Difficulté]
    Facile
  \end{moduleaction}
  \begin{moduleaction}[Priorités]
    \\\hline
    \begin{dndtable}
      \textbf{difficulté}  & \textbf{levels} \\
      easy      & 1 0 1 \\
      medium    & gnd vcc gnd \\
      hardcore  & 0x00 0x01 0x00 \\
    \end{dndtable}
  \end{moduleaction}
  \hline%
  \modulesection{Composants :}
  \begin{moduleaction}[fils]
    Je comporte pluieurs fils avec tout plein de couleurs.
  \end{moduleaction}
  \begin{moduleaction}[bouton]
    Un bouton de validation pour tester le raccord de fils.
  \end{moduleaction}
  \begin{moduleaction}[résistances]
    Mes résistances de retour sont nécessaire à la bonne lecture des pins.
  \end{moduleaction}
\end{modulebox}
\vspace{.5cm}

Il semblerait que vous devez brancher correctement les fils. Plutôt simple pour
celui-ci. On dit qu'une bonne communication est la clef du succès !
\newpage
