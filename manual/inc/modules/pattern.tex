\subsection{Patterns}
Des patterns ? Ah ouais ? Ah non des patrons. Quoi ? Des motifs ? Des schémas ?
Décidément, l'anglais technique c'est pas très clair ! Ici il va falloir
reconnaitre des \sout{patrons} motifs et appuyer sur les bons boutons.
\vspace{.5cm}
\begin{modulebox}{Pas terne}
  \begin{hangingpar}
    \begin{flushright}
    \textit{\dots Il fallait que je la fasse celle là}
    \end{flushright}
  \end{hangingpar}
  \modulesection{Description}
  \begin{moduleaction}[Difficulté]
    Pas si facile quand même
  \end{moduleaction}
  \begin{moduleaction}[Table de correspondance]
    \\\hline
    TODO
  \end{moduleaction}
  \hline%
  \modulesection{Composants :}
  \begin{moduleaction}[boutons]
    Il faut les appuyer pour valider un choix. Attention à ne pas se tromper.
  \end{moduleaction}
  \begin{moduleaction}[écran Arduino TFT]
    Un écran LCD rétroéclairé muni d'un emplacement pour carte SD. Avec ça, on
    peut générer tout un tas de formes géométriques et même charger des images.
  \end{moduleaction}
\end{modulebox}
\vspace{.5cm}

Écrire les directives de la résolution du module ici.
\newpage

