\subsection{Switch Mania}
Où il sera question de clic et de clac. Tâchez de trouver la bonne combinaisons
entre ces quatre interrupteurs ou ce ne sont pas les plombs qui vont sauter \ldots
mais vous !
\vspace{.5cm}
\begin{modulebox}{}
  \modulesection{Description}
  \begin{moduleaction}[Difficulté]
    On a connu plus dur
  \end{moduleaction}
  \begin{moduleaction}[But du jeu]
    Trouvez la bonne combinaison d'interrupteurs (ON/OFF) et appuyez sur le
    bouton pour la tester.
  \end{moduleaction}
  \modulesection{Composants :}
  \begin{moduleaction}[switch]
    De gros switchs ON/OFF qu'on retrouve parfois dans les avions. Ou au moins
    dans les films où y'a des avions.
  \end{moduleaction}
  \begin{moduleaction}[bouton]
    Vous pensez avoir trouvé ? Eh bien cliquez sur le bouton pour voir. Mais
    attention, la pénalité encourrue s'élève à $20 + difficulty * 5$ secondes.
  \end{moduleaction}
  \begin{moduleaction}[LED témoin]
    Elle vous indiquera si vous avez réussi à désamorcer le module.
  \end{moduleaction}
\end{modulebox}
\vspace{.5cm}

Vous avez devant vous quatre interrupteurs. Pour chacun des niveaux de
difficultés, il vous faudra entrer la bonne combinaison. Par exemple, une
combinaison $ON ON OFF ON$ pourrait être une combinaison valide pour le niveau
de difficulté intermédiaire (cet exemple est donné à titre d'exemple. Heu ?
C'est sûr ça ?). \\
Pour le niveau de difficulté le plus élevé, reportez-vous à la section
clic-clac. Pour ce qui est du niveau le plus bas, attendez un peu. Pour le
niveau intermédiaire, sautez la section suivante et lisez la prochaine.

\subsubsection{section suivante}
% facile : HIGH LOW HIGH HIGH
Bon ok on va éviter de tout faire péter. Je vais vous aider pas à pas. Suivez
bien mes instructions, c'est parti !
\begin{enumerate}
    % 1- L L L L
  \item Mettez tous les interrupteurs en position basse.
    % 2- L L H L
  \item $1 + 3 + 3 = 7$. $7$ étant un chiffre porte bonheur, vous devriez
        actionnez le troisième bouton.
    % 3- L H L H
  \item Inversez les états des deuxième et troisième interrupteurs.
    % 4- H L L H
  \item Inversez les états des deuxième et premier interrupteurs.
    % 5- H L H H
  \item Retournez à l'étape 3 et testez donc !
\end{enumerate}

\subsubsection{la prochaine}
% moyen : LOW HIGH HIGH LOW
\begin{figure}[H]
  \centering
  \includegraphics[width=.8\textwidth]{img/rebus}\hfill
\end{figure}

Indice : rebus are nice. \#FrenchTouch

\subsubsection{clic-clac}
% difficile : LOW LOW HIGH HIGH
\begin{center}
  clac clac clic clic BIP.
\end{center}
Indice : si c'est pas un clic, tentez un clac.

Si vous êtes arrivés jusque là en cherchant les instructions pour le niveau de
difficulté le plus bas. Tentez d'aller voir à la section suivante.
\newpage
